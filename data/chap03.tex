% !TeX root = ../thuthesis-example.tex

\chapter{基于扩散先验的4D Gaussian Splatting增强方法}

本章详细阐述了ReGen系统的核心方法论。本文方法旨在通过融合视频扩散模型的强大生成先验,来系统性地解决4D Gaussian Splatting(4DGS)在视点外推任务上的根本性局限。在深入具体方法前,本章首先介绍4DGS的基础理论框架,阐明其时间建模机制和渲染方程,然后从系统设计的整体视角介绍方法概览,最后深入各个技术组件的具体设计和实现细节。

\section{4D Gaussian Splatting基础理论}

\subsection{4D表示的数学含义}

4D Gaussian Splatting(4DGS)是3D Gaussian Splatting(3DGS)在时间维度上的扩展。"4D"指的是三维空间加上一维时间,即$\mathbb{R}^3 \times \mathbb{R}$。与静态的3DGS不同,4DGS需要建模场景中的动态变化,包括物体的运动、形状的变形以及外观的变化。

\textbf{4D高斯基元的参数化表示:}

一个4D高斯基元在时刻$t$可以用以下参数集合表示:

\begin{equation}
\mathcal{G}_{4D} = \left\{(\boldsymbol{\mu}_i(t), \mathbf{s}_i(t), \mathbf{q}_i(t), \alpha_i(t), \mathbf{c}_i(t, \mathbf{d}))\right\}_{i=1}^{N}
\label{eq:4dgs_parameterization}
\end{equation}

其中:
\begin{itemize}
\item $\boldsymbol{\mu}_i(t) \in \mathbb{R}^3$:时变的3D位置(中心点)
\item $\mathbf{s}_i(t) \in \mathbb{R}^3_+$:时变的尺度参数(椭球三轴长度)
\item $\mathbf{q}_i(t) \in \mathbb{S}^3$:时变的旋转四元数
\item $\alpha_i(t) \in [0,1]$:时变的不透明度
\item $\mathbf{c}_i(t, \mathbf{d}) \in \mathbb{R}^3$:时变的颜色,通常使用球谐函数(Spherical Harmonics)编码视角依赖性
\end{itemize}

\textbf{随时间变化的属性:}

不同于静态3DGS的固定参数,4DGS的所有参数都可能随时间变化:

\begin{itemize}
\item \textbf{位置轨迹} $\boldsymbol{\mu}_i(t)$:描述高斯基元在3D空间中的运动轨迹,对于动态物体(如车辆、行人)尤为重要
\item \textbf{尺度演化} $\mathbf{s}_i(t)$:反映物体尺寸的变化,例如车辆的接近或远离导致的视觉尺度变化
\item \textbf{旋转动态} $\mathbf{q}_i(t)$:捕捉物体的旋转运动,如转向的车辆或旋转的物体
\item \textbf{不透明度变化} $\alpha_i(t)$:建模物体的出现、消失以及遮挡关系的变化
\item \textbf{颜色变化} $\mathbf{c}_i(t, \mathbf{d})$:表示光照条件、材质反射等外观属性的时间演化
\end{itemize}

\textbf{静态与动态的分离表示:}

在自动驾驶场景中,通常将场景分解为静态背景和动态物体。静态背景的高斯参数不随时间变化(即$\frac{\partial}{\partial t} = 0$),而动态物体的高斯参数则需要显式建模时间依赖性。这种分离策略降低了模型复杂度,提高了训练效率。

\subsection{时间建模机制}

时间建模是4DGS的核心挑战。现有方法主要有以下几种时间表示策略:

\textbf{基于变形的方法:}

最直接的方式是对每个高斯基元的参数建模其相对于参考时刻(通常是$t=0$)的变形:

\begin{align}
\boldsymbol{\mu}_i(t) &= \boldsymbol{\mu}_i^{(0)} + \Delta\boldsymbol{\mu}_i(t) \label{eq:deformation_position} \\
\mathbf{s}_i(t) &= \mathbf{s}_i^{(0)} \odot \exp(\Delta\mathbf{s}_i(t)) \label{eq:deformation_scale} \\
\mathbf{q}_i(t) &= \mathbf{q}_i^{(0)} \otimes \Delta\mathbf{q}_i(t) \label{eq:deformation_rotation}
\end{align}

其中$\Delta\boldsymbol{\mu}_i(t)$, $\Delta\mathbf{s}_i(t)$, $\Delta\mathbf{q}_i(t)$分别是位置、尺度和旋转的变形量,可以通过时间嵌入网络或时间条件的MLP预测:

\begin{equation}
[\Delta\boldsymbol{\mu}_i(t), \Delta\mathbf{s}_i(t), \Delta\mathbf{q}_i(t)] = \mathcal{D}_{\text{deform}}(t, \boldsymbol{\theta}_i)
\label{eq:deformation_network}
\end{equation}

\textbf{基于轨迹的方法:}

对于物体级别的运动,可以显式建模其刚体变换轨迹。给定物体在时刻$t$的SE(3)变换$\mathbf{T}_{\text{obj}}(t) \in SE(3)$,物体上的高斯基元相应变换:

\begin{equation}
\boldsymbol{\mu}_i(t) = \mathbf{T}_{\text{obj}}(t) \circ \boldsymbol{\mu}_i^{\text{local}}
\label{eq:trajectory_based_position}
\end{equation}

物体轨迹$\mathbf{T}_{\text{obj}}(t)$可以通过跟踪算法获得,或使用参数化轨迹模型(如B样条、Bézier曲线)表示。

\textbf{基于特征的方法:}

使用学习的时间特征编码来隐式建模时间依赖性。每个时间步$t$被映射到一个特征向量$\mathbf{f}_t \in \mathbb{R}^d$,高斯参数通过该特征调制:

\begin{equation}
\boldsymbol{\mu}_i(t) = \boldsymbol{\mu}_i^{(0)} + \text{MLP}(\mathbf{f}_t, \boldsymbol{\theta}_i)
\label{eq:feature_based_modulation}
\end{equation}

\textbf{时间插值与外推:}

4DGS的一个关键能力是在未观测时刻进行插值($t \in [t_{\min}, t_{\max}]$)和外推($t < t_{\min}$ 或 $t > t_{\max}$)。基于变形的方法通常具有较好的插值能力,因为MLP可以学习平滑的时间函数。然而,外推能力受限于训练数据的时间范围。基于轨迹的方法通过显式的运动模型(如匀速、匀加速)可以实现一定程度的外推,但灵活性较低。

\subsection{4DGS渲染方程}

给定相机参数和时刻$t$,4DGS的渲染过程与3DGS类似,但需要在查询时刻对所有高斯参数进行求值。

\textbf{时间相关的渲染方程:}

对于像素$(u,v)$,其颜色由沿相机射线排序的高斯基元混合得到:

\begin{equation}
\mathbf{C}(u,v,t) = \sum_{i \in \mathcal{N}(u,v,t)} \mathbf{c}_i(t, \mathbf{d}) \alpha_i(t) T_i(u,v,t) G_i(u,v,t)
\label{eq:4dgs_rendering}
\end{equation}

其中:
\begin{itemize}
\item $\mathcal{N}(u,v,t)$:在时刻$t$对像素$(u,v)$有贡献的高斯基元集合
\item $\mathbf{c}_i(t, \mathbf{d})$:高斯基元$i$在时刻$t$沿视角方向$\mathbf{d}$的颜色
\item $\alpha_i(t)$:高斯基元$i$在时刻$t$的不透明度
\item $G_i(u,v,t)$:3D高斯函数在像素$(u,v)$处的投影值
\item $T_i(u,v,t)$:累积透射率
\end{itemize}

\textbf{3D高斯函数的投影:}

在时刻$t$,高斯基元$i$的3D高斯分布为:

\begin{equation}
G_i(\mathbf{x}, t) = \exp\left(-\frac{1}{2}(\mathbf{x} - \boldsymbol{\mu}_i(t))^T \Sigma_i^{-1}(t) (\mathbf{x} - \boldsymbol{\mu}_i(t))\right)
\label{eq:3d_gaussian}
\end{equation}

其中协方差矩阵$\Sigma_i(t)$由尺度和旋转参数构造:

\begin{equation}
\Sigma_i(t) = \mathbf{R}_i(t) \mathbf{S}_i(t) \mathbf{S}_i(t)^T \mathbf{R}_i(t)^T
\label{eq:covariance_matrix}
\end{equation}

其中$\mathbf{R}_i(t) \in SO(3)$是旋转矩阵(由四元数$\mathbf{q}_i(t)$转换得到),$\mathbf{S}_i(t) = \text{diag}(\mathbf{s}_i(t))$是尺度矩阵。

投影到2D图像平面时,使用EWA(Elliptical Weighted Average)splatting技术,得到2D高斯:

\begin{equation}
G_i(u,v,t) = \exp\left(-\frac{1}{2}\boldsymbol{\delta}_{i,t}^T \Sigma'_i{}^{-1}(t) \boldsymbol{\delta}_{i,t}\right)
\label{eq:2d_gaussian_projection}
\end{equation}

其中$\boldsymbol{\delta}_{i,t} = [(u,v) - \pi(\boldsymbol{\mu}_i(t))]^T$是像素位置到投影中心的偏移,$\pi(\cdot)$是3D到2D的投影函数,$\Sigma'_i(t)$是投影后的2D协方差矩阵。

\textbf{累积透射率的计算:}

累积透射率$T_i(u,v,t)$表示光线到达高斯基元$i$之前未被前面的高斯基元完全遮挡的概率:

\begin{equation}
T_i(u,v,t) = \prod_{j=1}^{i-1} (1 - \alpha_j(t) G_j(u,v,t))
\label{eq:transmittance}
\end{equation}

这要求高斯基元按照深度从近到远排序。在时刻$t$,排序依据是高斯中心$\boldsymbol{\mu}_i(t)$到相机的距离。

\textbf{可微性与时间梯度流:}

4DGS渲染是完全可微的,可以通过反向传播计算损失相对于所有高斯参数的梯度。关键的是,时间参数化使得梯度可以沿时间流动,从而学习时间一致的动态表示。时间梯度为:

\begin{equation}
\frac{\partial \mathcal{L}}{\partial \boldsymbol{\mu}_i^{(0)}} = \sum_{t \in \mathcal{T}} \frac{\partial \mathcal{L}(t)}{\partial \mathbf{C}(t)} \frac{\partial \mathbf{C}(t)}{\partial \boldsymbol{\mu}_i(t)} \frac{\partial \boldsymbol{\mu}_i(t)}{\partial \boldsymbol{\mu}_i^{(0)}}
\label{eq:temporal_gradient}
\end{equation}

这种时间梯度流确保了学习到的动态模型在时间上的平滑性和一致性。

理解4DGS的基础理论后,我们现在转向ReGen系统的具体设计。在下面的章节中,我们将介绍如何通过视频扩散模型生成高质量监督信号来增强4DGS的新视点泛化能力。

\section{方法概览与设计理念}
\label{sec:method_overview}

\subsection{技术演进与方法定位}

在介绍ReGen的具体设计之前,有必要阐述本文方法的形成过程和技术定位。研究工作经历了从NeRF到3DGS,再到基于扩散模型的4DGS的技术演进,每一阶段都针对前一阶段的核心局限进行了系统性改进。

NeRF的连续函数表示理论上可以提供任意分辨率的渲染结果,这对于需要多尺度观察的自动驾驶仿真具有吸引力。然而,实际应用中暴露出三个根本性局限。首先是计算复杂度问题,每个像素的渲染需要沿光线进行密集采样并执行神经网络前向传播,单帧渲染耗时较长,与实时仿真要求存在显著差距。其次是可扩展性限制,单一MLP网络的表达容量有限,难以编码跨越数公里范围的城市级场景。即使采用Block-NeRF分块策略,也面临模型管理和边界衔接的复杂性。最后是动态建模的困难,NeRF的静态场景假设与自动驾驶环境的动态本质存在矛盾,基于形变场的扩展方法难以处理多物体独立运动和物体出现消失的情况。这些局限性促使研究者们转向更高效的显式表示方法。

3D Gaussian Splatting的显式点云表示带来了实质性突破,实现了实时渲染性能。通过动静分离策略,该方法能够有效处理动态驾驶场景重建。然而,3DGS在视点外推任务中暴露出严重缺陷。从训练数据未覆盖的新视角(如换道轨迹、俯视视角)渲染时,图像质量急剧下降,出现几何扭曲、物体缺失和视觉伪影。深入分析表明,这些问题源于三个方面:优化目标导致高斯基元排布特化于训练视角,缺乏全局几何一致性约束;离散点云在空间稀疏区域缺乏有效补全机制;纯数据驱动学习缺乏对场景结构的内在理解。这些分析表明,单纯依靠几何优化无法解决泛化问题,需要引入强大的生成先验知识。

扩散模型在大规模预训练中内隐学习了丰富的视觉先验,包括场景结构合理性、物体外观一致性和光照真实性。关键思路是利用扩散模型为未覆盖视角生成高质量"伪真值"作为监督信号,引导4DGS训练。核心挑战在于几何可控性——无条件生成会导致几何不可控。因此本文采用LiDAR几何作为条件:LiDAR提供稀疏但准确的3D骨架,扩散模型在此基础上填充视觉细节。基于扩散模型的4DGS框架具有三个关键设计特征。在几何条件编码方面,将LiDAR点云渲染为深度图并编码为RGB通道,既保留几何信息又与扩散模型输入格式兼容。在训练推理解耦方面,扩散模型仅在训练阶段生成监督信号,推理时完全依赖已优化的4DGS,实现了先验增强与部署效率的平衡。在自适应监督策略方面,通过训练自由引导和自适应采样调度,实现监督强度随训练进程动态调整,有效避免了训练不稳定的问题。

\begin{table}[htbp]
\centering
\caption{技术路线演进对比}
\label{tab:technical-evolution}
\small
\begin{tabular}{p{2.2cm}p{3.2cm}p{4cm}p{2.8cm}}
\toprule
\textbf{技术方法} & \textbf{核心优势} & \textbf{主要局限} & \textbf{关键发现} \\
\midrule
NeRF & 连续表示,质量高 & 渲染慢;扩展性差;动态建模难 & 不适合实时动态场景 \\
\midrule
3DGS & 实时渲染,有效动静分离 & 视点外推弱;几何扭曲;缺乏先验 & 需引入生成先验 \\
\midrule
3DGS + Diffusion & 初步引入先验 & 效果有限;不稳定 & 需系统性设计 \\
\midrule
\textbf{Diffusion-based 4DGS} & \textbf{先验增强+高效渲染} & \textbf{两阶段训练} & \textbf{先验与效率平衡} \\
\bottomrule
\end{tabular}
\end{table}

这一技术演进表明,高保真场景重建需要同时解决效率和泛化两个核心问题。本文方法通过扩散监督框架实现了生成先验与渲染效率的有机融合:扩散模型在训练阶段提供高质量监督信号,4DGS在推理阶段实现实时高效渲染。这一设计为理解后续具体方法提供了必要的技术背景。

\subsection{整体框架设计}

ReGen的核心设计理念是利用强大但计算昂贵的视频扩散模型生成高质量监督信号,引导高效但泛化能力有限的4D Gaussian Splatting进行优化,实现两者的优势互补。整个框架采用两阶段训练策略。在第一阶段,在大规模驾驶数据上训练一个以LiDAR几何条件为引导的视频扩散模型,该模型学习从稀疏的几何信息(LiDAR点云渲染)生成高质量、时间连贯的驾驶视频。在第二阶段,利用预训练的扩散模型为新视点生成高质量的"伪真值"图像,以此作为监督信号指导4DGS模型的训练,从而提升其视点外推能力。

这种设计的核心优势体现在三个方面。首先,扩散模型通过大规模预训练获得了强大的视觉先验和生成能力,能够为未见视点提供合理的场景内容补全。其次,4DGS保持了实时渲染的计算效率,确保系统在推理阶段的实用性。最后,扩散生成的监督信号有效桥接了两者,使得系统在训练阶段充分利用生成先验,而在部署阶段保持高效渲染,实现了生成质量与计算效率的最优平衡。

\subsection{数据流与训练策略概览}

整个训练过程涉及多种类型的数据和处理流程。系统的输入数据主要包括五个方面:来自多视角相机的RGB图像序列提供视觉真值;64线激光雷达采集的LiDAR点云提供几何约束和深度监督;3D边界框序列描述物体轨迹,用于动静分离和物体跟踪;相机标定数据包含内外参数,确保几何一致性;天空掩码通过Grounding DINO + SAM生成,用于天空区域的特殊处理。

训练数据流程遵循一个清晰的五步处理链条。首先,将LiDAR点云投影并渲染为RGB-D条件图像,作为扩散模型的几何条件。随后,使用RGB图像和LiDAR条件训练可控视频生成模型,使其学习从几何骨架到完整视觉场景的映射关系。在扩散模型训练完成后,为训练场景创建虚拟的换道、转弯等新视点轨迹,这些轨迹偏离原始训练视点分布,模拟实际应用中的视点外推需求。接下来,扩散模型基于新轨迹的LiDAR条件生成高质量视频,作为"伪真值"监督信号。最后,在4DGS的优化过程中同时使用真实图像和扩散生成的伪真值进行联合训练,通过这种方式将扩散模型的生成先验有效迁移到4DGS中,提升其在未见视点上的泛化能力。

\section{几何与条件控制}

本节介绍系统中几何信息的利用和条件控制机制,包括LiDAR数据的多层次处理、天空区域的智能分割,以及可控视频扩散模型的设计。

\subsection{LiDAR数据的多层次利用}

LiDAR数据在我们的系统中发挥了三个关键作用:几何条件生成、深度监督和动静分离。这种多层次利用确保了几何一致性和训练稳定性。

\textbf{几何条件生成:}
我们将LiDAR点云渲染为像素级的几何条件,为扩散模型提供精确的3D结构约束。具体流程包括:

\begin{enumerate}
\item \textbf{点云聚合}:聚合$\pm\Delta t$帧内的LiDAR点云以增加密度
\item \textbf{点云着色}:通过相机图像为点云添加RGB信息
\item \textbf{可微分渲染}:将彩色点云渲染为RGB-D条件图像
\end{enumerate}

数学上,给定时刻$t$的LiDAR点云$\mathcal{P}_t = \{\mathbf{p}_i, \mathbf{c}_i\}_{i=1}^{N_t}$,我们首先进行时间聚合:

\begin{equation}
\mathcal{P}_{\text{agg}}(t) = \bigcup_{\tau=t-\Delta t}^{t+\Delta t} \mathcal{T}_{\text{ego}}(\tau \rightarrow t) \circ \mathcal{P}_\tau
\label{eq:lidar_aggregation}
\end{equation}

其中$\mathcal{T}_{\text{ego}}(\tau \rightarrow t)$是自车从时刻$\tau$到$t$的位姿变换。

然后通过可微分点渲染生成条件图像:

\begin{equation}
\mathbf{C}_{\text{lidar}}(u,v) = \sum_{i: \pi(\mathbf{p}_i) = (u,v)} w_i \cdot [\mathbf{c}_i; d_i; m_i]
\label{eq:lidar_condition_generation}
\end{equation}

其中$\pi(\cdot)$是3D到2D的投影函数,$w_i$是可见性权重,$d_i$是归一化深度,$m_i$是有效性掩码。

\textbf{深度监督:}
LiDAR提供的稀疏深度用于约束4DGS的几何重建:

\begin{equation}
\mathcal{L}_{\text{depth}} = \frac{1}{|\mathcal{M}_{\text{valid}}|} \sum_{(u,v) \in \mathcal{M}_{\text{valid}}} \rho(D_{\text{render}}(u,v) - D_{\text{lidar}}(u,v))
\label{eq:depth_supervision}
\end{equation}

其中$\mathcal{M}_{\text{valid}}$是有效LiDAR深度像素,$\rho(\cdot)$是Huber损失函数。

\subsection{动静分离的LiDAR处理策略}

为了支持动态场景建模,我们对LiDAR点云进行动静分离处理:

\textbf{静态背景点云:}
通过物体轨迹信息,将不在任何3D边界框内的点云归类为静态背景:

\begin{equation}
\mathcal{P}_{\text{bkgd}}(t) = \{\mathbf{p}_i \in \mathcal{P}_t : \forall j, \mathbf{p}_i \notin \mathcal{B}_j(t)\}
\label{eq:background_points}
\end{equation}

其中$\mathcal{B}_j(t)$是第$j$个物体在时刻$t$的3D边界框。

对于动态物体,每个跟踪物体的点云单独处理,提取属于特定物体边界框$\mathcal{B}_j(t)$内的所有点:

\begin{equation}
\mathcal{P}_{\text{obj}}^{(j)}(t) = \{\mathbf{p}_i \in \mathcal{P}_t : \mathbf{p}_i \in \mathcal{B}_j(t)\}
\label{eq:object_points}
\end{equation}

这种动静分离策略具有三个关键优势。首先,静态背景可以跨时间聚合,显著提高点云密度,从而获得更加完整的几何表示。其次,动态物体保持时间独立性,避免了时间聚合带来的运动模糊问题。最后,这种分离策略支持物体级别的编辑和控制,为后续的场景操作提供了灵活性。

\subsection{天空区域的智能处理}

天空区域在自动驾驶场景中具有特殊性:它缺乏明确的几何结构,且在不同视点下变化复杂。我们采用了基于视觉语言模型的智能分割策略,该策略采用两阶段流程实现准确的天空区域分割。首先使用Grounding DINO通过文本提示"sky"检测天空区域的粗略边界框,然后利用SAM(Segment Anything Model)基于检测框生成精确的像素级天空掩码。

为了确保天空检测的准确性,我们在具体实现中引入了空间约束策略:

\begin{figure}[!b]
  \begin{tmpbox}
    \lstset{
      basicstyle=\fontsize{10}{11}\ttfamily, 
      numbersep=5pt, 
      columns=flexible, 
      captionpos=t,
      numbers=left,
      numberstyle=\tiny\ttfamily,
      breaklines=true,
    }
    \begin{lstlisting}[language=Python]
# Sky detection with spatial constraints
boxes_mask = boxes_xyxy[:, 1] < 100  # Top 100 pixels only
valid_boxes = boxes_xyxy[boxes_mask]

# SAM segmentation with box prompts
sam_predictor.set_image(image_source)
transformed_boxes = sam_predictor.transform.apply_boxes_torch(valid_boxes, image_source.shape[:2])
masks, _, _ = sam_predictor.predict_torch(
    point_coords=None,
    point_labels=None, 
    boxes=transformed_boxes,
    multimask_output=False
)
\end{lstlisting}
  \end{tmpbox}
  \caption{天空检测实现}
  \label{fig:sky-detection-code}
\end{figure}

我们针对不同场景特性设计了三种天空表示策略。高斯天空模型使用独立的高斯基元集合表示天空,适用于天空区域较大且需要精细建模的场景。立方体贴图策略使用环境贴图表示天空,计算效率高但表达能力相对有限,适合快速渲染场景。背景集成策略将天空直接集成到背景高斯模型中,适用于天空区域较小或天空不是主要视觉关注点的场景。

针对天空区域的特殊性质,我们设计了基于不透明度分布的天空损失函数:

\begin{equation}
\mathcal{L}_{\text{sky}} = \mathbb{E}_{(u,v) \in \mathcal{M}_{\text{sky}}} [-\log(1-\alpha(u,v))] + \mathbb{E}_{(u,v) \in \mathcal{M}_{\text{non-sky}}} [H(\alpha(u,v))]
\label{eq:sky_loss_design}
\end{equation}

其中$H(\alpha) = -\alpha \log \alpha - (1-\alpha) \log(1-\alpha)$是二元熵函数。这种设计的核心思想是鼓励天空区域呈现低不透明度($\alpha \rightarrow 0$),使得天空颜色能够透过高斯基元直接显示,同时促使非天空区域保持确定性的不透明度分布($\alpha \rightarrow 0$ 或 $\alpha \rightarrow 1$),从而在几何建模中实现清晰的前景与背景分离。

\subsection{可控视频扩散模型设计}

我们的视频扩散模型采用了多层次的条件控制机制,确保生成内容的几何一致性和视觉质量。该机制包含三个层次的条件输入:像素级几何条件通过LiDAR渲染的RGB-D图像提供逐像素的几何约束,全局场景条件通过相机参数和场景元数据提供全局上下文信息,时间条件通过帧索引和时间戳确保生成视频的时间连贯性。

这些多层次条件的编码过程可以表示为:

\begin{align}
\mathbf{h}_{\text{pixel}} &= \text{Conv2D}(\mathbf{C}_{\text{lidar}}) \\
\mathbf{h}_{\text{global}} &= \text{MLP}([\mathbf{K}; \mathbf{T}; t]) \\
\mathbf{h}_{\text{cond}} &= \text{CrossAttention}(\mathbf{h}_{\text{pixel}}, \mathbf{h}_{\text{global}})
\label{eq:condition_encoding_multilevel}
\end{align}

\begin{figure}[htbp]
  \centering
  \includegraphics[width=0.95\textwidth]{pdf/diffusion_architecture.pdf}
  \caption{可控视频扩散模型架构}
  \label{fig:diffusion-architecture}
\end{figure}

为了确保生成视频的时间连贯性,我们设计了分解式时空注意力机制:

\begin{equation}
\text{SpatioTemporalAttention}(\mathbf{X}) = \text{TemporalAttention}(\text{SpatialAttention}(\mathbf{X}))
\label{eq:factorized_attention}
\end{equation}

这种分解设计具有显著的优势。首先,计算复杂度从完全时空注意力的$O(T^2H^2W^2)$降低到$O(TH^2W^2 + T^2HW)$,使得模型能够处理更长的视频序列。其次,空间注意力和时间注意力的分离使得模型能够更好地建模空间和时间的不同特性,空间注意力专注于帧内的局部结构,时间注意力专注于跨帧的运动模式。最后,这种设计支持不同分辨率的高效处理,便于在不同计算资源约束下进行灵活部署。

\section{训练流程与扩散监督}

本节详细介绍分阶段训练策略、训练自由引导机制以及新视点轨迹的生成方法。

\begin{figure}[htbp]
  \centering
  \includegraphics[width=0.95\textwidth]{pdf/training_pipeline.pdf}
  \caption{ReGen训练流程序列图}
  \label{fig:training-pipeline}
\end{figure}

\subsection{分阶段训练策略}

我们的训练策略采用了精心设计的分阶段方法,每个阶段都有明确的目标和数据需求。

在第一阶段的扩散模型预训练中,使用RGB视频序列$\mathbf{X} = \{\mathbf{x}_t\}_{t=1}^{T}$、RGB-D格式的LiDAR条件$\mathbf{C} = \{\mathbf{c}_t\}_{t=1}^{T}$,以及确保几何一致性的相机内外参数进行若干轮训练。训练目标采用标准的去噪扩散概率模型损失函数:

\begin{equation}
\mathcal{L}_{\text{diffusion}} = \mathbb{E}_{t,\epsilon,\mathbf{x}_0,\mathbf{C}} \left[ \|\epsilon - \epsilon_\theta(\sqrt{\bar{\alpha}_t} \mathbf{x}_0 + \sqrt{1-\bar{\alpha}_t} \epsilon, \mathbf{C}, t)\|_2^2 \right]
\label{eq:diffusion_pretraining}
\end{equation}

该阶段采用的关键技术包括:以适当概率进行条件丢弃以支持分类器自由引导,应用时空注意力机制确保生成视频的时间连贯性,以及采用多尺度训练策略支持不同分辨率的生成需求。

在第二阶段的4DGS基础训练中,初期阶段专注于建立稳定的几何表示。该阶段使用训练视点$\mathcal{V}_{\text{train}}$(包含相机参数和真实图像)、用于高斯基元初始化的彩色点云、用于动静分离的物体轨迹,以及提供几何监督的LiDAR深度数据。训练目标结合了重建损失、深度监督和天空区域约束:

\begin{equation}
\mathcal{L}_{\text{base}} = \mathcal{L}_{\text{recon}} + \lambda_{\text{depth}} \mathcal{L}_{\text{depth}} + \lambda_{\text{sky}} \mathcal{L}_{\text{sky}}
\label{eq:base_training}
\end{equation}

该阶段的关键操作包括:周期性地基于梯度和几何特征进行自适应密化,逐步递增球谐函数阶数以提升外观建模能力,定期重置低质量高斯基元以维持模型质量。

在第三阶段的扩散监督训练中,系统同时使用原始训练数据(继续进行真实图像的重建训练)、虚拟的换道和转弯等新视点轨迹、扩散模型生成的高质量伪真值作为新视点监督信号,以及用于掩码引导和损失计算的天空掩码。训练目标在重建损失的基础上增加了新视点损失和正则化项:

\begin{equation}
\mathcal{L}_{\text{train}} = \mathcal{L}_{\text{recon}} + \lambda_{\text{novel}} \mathcal{L}_{\text{novel}} + \lambda_{\text{reg}} \mathcal{L}_{\text{reg}}
\label{eq:diffusion_supervised_training}
\end{equation}

该阶段采用的关键策略包括:在训练的关键节点进行渐进式扩散采样,引导强度从初始的强引导逐渐衰减到后期的弱引导,以及在训练后期启用掩码引导机制,仅在渲染质量较差的区域应用扩散监督以提高训练效率。

\begin{figure}[htbp]
  \centering
  \includegraphics[width=0.95\textwidth]{pdf/knowledge_distillation.pdf}
  \caption{扩散监督框架}
  \label{fig:diffusion-supervision}
\end{figure}

\subsection{训练自由引导机制}

我们提出的训练自由引导机制是ReGen的核心创新之一。与传统的分类器自由引导不同,该机制同时使用几何条件和当前4DGS渲染结果作为双重引导信号:

\begin{equation}
\epsilon_{\text{guided}} = \epsilon_{\text{uncond}} + w_{\text{geom}} \cdot (\epsilon_{\text{geom}} - \epsilon_{\text{uncond}}) + w_{\text{render}} \cdot (\epsilon_{\text{render}} - \epsilon_{\text{uncond}})
\label{eq:dual_guidance}
\end{equation}

其中$\epsilon_{\text{geom}}$表示基于LiDAR几何条件的噪声预测,$\epsilon_{\text{render}}$表示基于当前4DGS渲染结果的噪声预测,$w_{\text{geom}}$和$w_{\text{render}}$是自适应调整的引导权重。这两个权重在训练过程中动态调整,以平衡几何约束和视觉保真度:

\begin{align}
w_{\text{geom}}(k) &= w_{\text{geom}}^{\text{init}} \cdot (1 + \cos(\pi k / K)) / 2 \\
w_{\text{render}}(k) &= w_{\text{render}}^{\text{init}} \cdot \exp(-k / \tau_{\text{decay}})
\label{eq:adaptive_guidance_weights}
\end{align}

这种自适应权重设计确保了训练初期保持强几何约束以建立正确的空间结构,而在训练后期转向精细调优以提升视觉质量。

在训练后期,我们进一步引入掩码引导机制,通过评估当前渲染质量选择性地应用扩散监督,仅在渲染质量较差的区域进行强化,从而提高训练效率并避免过度监督导致的模型退化:

\begin{equation}
\mathbf{M}_{\text{guide}}(u,v) = \begin{cases}
1, & \text{if } \|\mathbf{I}_{\text{render}}(u,v) - \mathbf{I}_{\text{ref}}(u,v)\|_2 > \tau_{\text{error}} \\
0, & \text{otherwise}
\end{cases}
\label{eq:masked_guidance}
\end{equation}

这种策略的优势:
- 避免在已经收敛的区域进行不必要的修正
- 集中计算资源在最需要改进的区域
- 防止过度拟合扩散模型的生成偏好

\subsection{新视点轨迹的生成策略}

为了训练4DGS的视点外推能力,我们需要生成多样化的新视点轨迹。这些轨迹模拟真实驾驶中的各种机动行为:

\textbf{横向平移轨迹}:模拟换道行为
\begin{equation}
\mathbf{T}_{\text{lateral}}(t) = \mathbf{T}_{\text{orig}}(t) \cdot \text{Translate}(d_{\text{lateral}} \cdot \mathbf{e}_y, 0, 0)
\label{eq:lateral_trajectory}
\end{equation}

\textbf{纵向偏移轨迹}:模拟跟车距离变化
\begin{equation}
\mathbf{T}_{\text{longitudinal}}(t) = \mathbf{T}_{\text{orig}}(t) \cdot \text{Translate}(0, d_{\text{longitudinal}}, 0)
\label{eq:longitudinal_trajectory}
\end{equation}

\textbf{高度变化轨迹}:模拟不同车辆高度
\begin{equation}
\mathbf{T}_{\text{height}}(t) = \mathbf{T}_{\text{orig}}(t) \cdot \text{Translate}(0, 0, d_{\text{height}})
\label{eq:height_trajectory}
\end{equation}

这些新轨迹的LiDAR条件通过几何变换自动生成,确保了条件的几何一致性。

\section{4DGS优化与正则化}

本节介绍多模态损失函数设计、自适应密化策略以及多种正则化约束,确保4DGS模型的重建质量和几何一致性。

\subsection{多模态损失函数的协同设计}

我们设计了一个综合的多模态损失函数,平衡重建质量、几何一致性和新视点泛化:

\begin{align}
\mathcal{L}_{\text{total}} &= \mathcal{L}_{\text{recon}} + \lambda_{\text{novel}} \mathcal{L}_{\text{novel}} + \lambda_{\text{depth}} \mathcal{L}_{\text{depth}} \\
&\quad + \lambda_{\text{scale}} \mathcal{L}_{\text{scale}} + \lambda_{\text{sky}} \mathcal{L}_{\text{sky}} + \lambda_{\text{temporal}} \mathcal{L}_{\text{temporal}}
\label{eq:comprehensive_loss}
\end{align}

\textbf{重建损失}:多尺度感知损失组合
\begin{align}
\mathcal{L}_{\text{recon}} &= \mathcal{L}_{\text{L1}} + \lambda_{\text{ssim}} \mathcal{L}_{\text{SSIM}} + \lambda_{\text{lpips}} \mathcal{L}_{\text{LPIPS}} \\
\mathcal{L}_{\text{L1}} &= \|\mathbf{M} \odot (\mathbf{I}_{\text{render}} - \mathbf{I}_{\text{gt}})\|_1 \\
\mathcal{L}_{\text{SSIM}} &= 1 - \text{SSIM}(\mathbf{I}_{\text{render}}, \mathbf{I}_{\text{gt}}, \mathbf{M}) \\
\mathcal{L}_{\text{LPIPS}} &= \text{LPIPS}(\mathbf{M} \odot \mathbf{I}_{\text{render}}, \mathbf{M} \odot \mathbf{I}_{\text{gt}})
\label{eq:reconstruction_loss_detailed}
\end{align}

\textbf{新视点监督损失}:基于扩散生成的伪真值
\begin{equation}
\mathcal{L}_{\text{novel}} = \|\mathbf{M}_{\text{upper}} \odot (\mathbf{I}_{\text{render}}^{\text{novel}} - \mathbf{I}_{\text{pseudo}})\|_1 + \lambda_{\text{novel\_ssim}} (1 - \text{SSIM}(\mathbf{I}_{\text{render}}^{\text{novel}}, \mathbf{I}_{\text{pseudo}}, \mathbf{M}_{\text{upper}}))
\label{eq:novel_supervision_loss}
\end{equation}

其中$\mathbf{M}_{\text{upper}}$排除图像上半部分,避免天空区域的不稳定性。

\textbf{时间一致性损失}:确保动态物体的平滑运动
\begin{equation}
\mathcal{L}_{\text{temporal}} = \sum_{i=1}^{N_{\text{obj}}} \sum_{t=1}^{T-1} \|\mathbf{T}_i(t+1) - \mathbf{T}_i(t)\|_{\text{SE(3)}}^2
\label{eq:temporal_consistency}
\end{equation}

\subsection{自适应密化的智能策略}

传统3DGS的密化策略在动态场景中面临挑战。我们提出了针对动态场景的自适应密化策略:

\begin{figure}[htbp]
  \centering
  \includegraphics[width=0.95\textwidth]{pdf/gaussian_densification.pdf}
  \caption{自适应高斯密化算法流程}
  \label{fig:gaussian-densification}
\end{figure}

\begin{algorithm}
\caption{4D高斯自适应密化算法}
\label{alg:adaptive_densification}
\begin{algorithmic}[1]
\REQUIRE 高斯集合$\mathcal{G}$,观测序列$\mathcal{O}$
\ENSURE 优化后的高斯集合$\mathcal{G}^*$

\STATE \textbf{阶段1: 质量评估}
\FOR{每个高斯基元 $\mathcal{G}_i \in \mathcal{G}$}
    \STATE 计算重建质量指标:$\mathbf{Q}_i = f_{\text{eval}}(\mathcal{G}_i, \mathcal{O})$
    \STATE 评估几何合理性:$\mathbf{V}_i = g_{\text{geom}}(\mathcal{G}_i)$
\ENDFOR

\STATE \textbf{阶段2: 密化操作}
\FOR{每个高斯基元 $\mathcal{G}_i$}
    \IF{$\mathbf{Q}_i$ 低于质量阈值}
        \IF{满足分裂条件}
            \STATE 执行分裂操作生成子高斯
        \ELSE
            \STATE 执行复制操作增加密度
        \ENDIF
    \ENDIF
\ENDFOR

\STATE \textbf{阶段3: 剪枝优化}
\FOR{每个高斯基元 $\mathcal{G}_i$}
    \IF{$\mathbf{V}_i$ 不满足几何约束 OR 贡献度过低}
        \STATE 从集合中移除该高斯基元
    \ENDIF
\ENDFOR

\RETURN $\mathcal{G}^*$
\end{algorithmic}
\end{algorithm}

\subsection{几何一致性约束}

几何一致性约束通过多种方式确保4DGS重建的准确性和合理性。LiDAR深度约束利用激光雷达提供的稀疏但准确的深度信息监督4DGS的几何重建质量:

\begin{equation}
\mathcal{L}_{\text{depth}} = \frac{1}{|\mathcal{M}_{\text{lidar}}|} \sum_{(u,v) \in \mathcal{M}_{\text{lidar}}} \text{SmoothL1}(D_{\text{render}}(u,v), D_{\text{lidar}}(u,v))
\label{eq:lidar_depth_constraint}
\end{equation}

物体边界约束进一步确保动态物体的高斯基元保持在其对应边界框内,避免几何泄漏导致的重建伪影:

\begin{equation}
\mathcal{L}_{\text{bbox}} = \sum_{i=1}^{N_{\text{obj}}} \mathbb{E}_{\mathbf{p} \notin \mathcal{B}_i} [\alpha_i(\mathbf{p})] + \mathbb{E}_{\mathbf{p} \in \mathcal{B}_i} [H(\alpha_i(\mathbf{p}))]
\label{eq:bbox_constraint}
\end{equation}

\subsection{尺度和形状正则化}

尺度和形状正则化通过约束高斯基元的几何参数来提升重建质量。各向异性约束防止高斯基元退化为过度拉伸的椭球体,通过惩罚尺度参数的极端不均匀性来保持合理的形状分布:

\begin{equation}
\mathcal{L}_{\text{anisotropy}} = \mathbb{E}_{g \in \mathcal{G}} \left[ \frac{\max(\mathbf{s}_g)}{\min(\mathbf{s}_g)} - 1 \right]^2
\label{eq:anisotropy_regularization}
\end{equation}

紧凑性约束则鼓励高斯基元保持紧凑的形状,避免占据过大的空间体积,从而提高场景表示的精确性:

\begin{equation}
\mathcal{L}_{\text{compact}} = \mathbb{E}_{g \in \mathcal{G}} \left[ \|\mathbf{s}_g\|_2^2 \right]
\label{eq:compactness_regularization}
\end{equation}

\section{数据依赖与训练调度}

本节分析系统训练的完整数据依赖关系和精心设计的训练调度策略。

\subsection{数据流向分析}

整个训练过程的数据依赖关系呈现为多层次的处理流程。首先,原始数据经过预处理阶段,将TFRecord格式的多模态数据转换为标准化的训练输入。具体而言,RGB图像序列、LiDAR点云、物体轨迹和相机标定参数从原始数据中提取,随后通过Grounding DINO和SAM模型生成天空掩码,并将LiDAR点云与RGB图像融合形成彩色点云,最终结合相机参数生成LiDAR条件图像(RGB-D格式)。

在预处理数据向训练数据的转换过程中,RGB序列与LiDAR条件结合形成扩散模型的训练对,彩色点云与物体轨迹共同用于4DGS的初始化,原始轨迹经过几何变换生成新视点轨迹,这些新视点轨迹与LiDAR条件一起构成扩散采样的输入。训练数据最终转化为多种监督信号:真实RGB图像提供重建损失监督,LiDAR深度信息确保几何一致性,扩散生成图像为新视点提供监督信号,天空掩码则专门用于天空区域的损失监督。

\subsection{训练调度的精确设计}

我们的训练调度采用了精心设计的多阶段策略,确保模型能够稳定收敛并逐步提升性能。在预热阶段,系统仅使用重建损失来建立基础几何结构,采用较低的学习率,并禁用密化操作以避免早期训练的不稳定性。

基础训练阶段启用完整的重建损失和正则化项,实施自适应密化策略,周期性地进行密化操作。同时,球谐函数的阶数采用递增策略,逐步提升表示能力。

扩散监督阶段在训练的关键节点进行扩散采样,采用从强到弱渐进式调整的引导强度,新视点采样以适当概率进行。最后,在掩码引导阶段,系统启用掩码引导进行精细化训练,降低密化频率以稳定几何结构,并引入时间一致性约束。

\section{理论分析与性能保证}

本节从理论角度分析方法的收敛性和计算复杂度。

\subsection{收敛性的理论保证}

\begin{theorem}[多模态损失收敛性]
在以下条件下:
\begin{enumerate}
\item 扩散模型生成质量有界:$\mathbb{E}[\|\mathcal{D}_\phi(\mathbf{z}, \mathbf{C}) - \mathbf{I}_{\text{true}}\|_2] \leq \epsilon_D$
\item 4DGS表示能力充分:存在$\mathcal{G}^*$使得表示误差有界
\item 学习率满足标准条件:$\sum \eta_k = \infty$,$\sum \eta_k^2 < \infty$
\end{enumerate}
优化过程几乎必然收敛到$\epsilon_D$-邻域内的最优解:
\begin{equation}
\lim_{k \to \infty} \mathbb{E}[\mathcal{L}_{\text{total}}^{(k)}] \leq \mathcal{L}^* + C \cdot \epsilon_D
\end{equation}
\end{theorem}

\subsection{计算复杂度的全面分析}

系统的计算复杂度主要来源于扩散模型和4DGS两个核心组件。扩散模型的预训练阶段复杂度为$O(E \cdot B \cdot T \cdot K \cdot H \cdot W \cdot D_{\text{model}})$,其中$E$表示训练轮数,$B$为批次大小,$T$为时间步数,$K$为去噪步数,$H \times W$为图像分辨率,$D_{\text{model}}$为模型维度。推理采样阶段的复杂度为$O(K \cdot T \cdot H \cdot W \cdot D_{\text{model}})$,相比预训练阶段显著降低。

4DGS组件的计算复杂度主要体现在渲染、密化和梯度计算三个环节。渲染操作的复杂度为$O(N_g \cdot H \cdot W \cdot \log N_g)$,其中$N_g$为高斯数量,对数项来源于排序操作。密化操作的复杂度为$O(N_g \cdot D_{\text{gaussian}})$,梯度计算的复杂度为$O(N_g \cdot H \cdot W)$。

综合上述分析,系统的总体训练复杂度可表示为:
\begin{equation}
\mathcal{O}_{\text{total}} = O(I \cdot (N_g \cdot H \cdot W + N_{\text{sample}} \cdot K \cdot T \cdot H \cdot W \cdot D_{\text{model}}))
\label{eq:total_complexity}
\end{equation}
其中$I$是总迭代数,$N_{\text{sample}}$是扩散采样频率。该复杂度表达式表明,系统的主要计算开销来自于4DGS的渲染操作和扩散模型的采样过程。

\section{本章小结}

本章从4DGS的基础理论出发,系统性地阐述了ReGen的核心方法论。本文首先介绍了4D Gaussian Splatting的数学表示、时间建模机制和渲染方程,为理解后续方法奠定了理论基础。

在方法设计层面,我们提出了双阶段扩散监督框架,利用视频扩散模型生成高质量监督信号来引导4DGS的优化训练。通过LiDAR几何条件的多层次利用、智能天空处理和精心设计的分阶段训练策略,系统实现了几何一致性和视觉质量的平衡。

训练自由引导机制是我们的核心创新,通过双重引导和掩码控制实现了精确的监督信号传递。多模态损失函数的协同设计确保了重建质量、几何一致性和新视点泛化能力的平衡。自适应密化算法针对动态场景进行了专门优化,支持时间感知的几何表示。

理论分析证明了方法的收敛性和复杂度特性,为实际应用提供了理论指导。该方法论框架为解决4DGS在视点外推任务上的挑战提供了系统性的解决方案,下一章将详细介绍具体的工程实现和系统架构。
