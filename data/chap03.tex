% !TeX root = ../thuthesis-example.tex

\chapter{基于扩散先验的4D高斯溅射增强方法}

本章详细阐述了StreetCrafter系统的核心方法论。我们的方法旨在通过融合视频扩散模型的强大生成先验,来系统性地解决4D高斯溅射在视点外推任务上的根本性局限。本章首先从系统设计的整体视角介绍方法概览,然后深入各个技术组件的具体设计和实现细节。

\section{方法概览与设计理念}

\subsection{整体框架设计}

StreetCrafter的核心设计理念是将强大但计算昂贵的视频扩散模型作为"教师",将高效但泛化能力有限的4D高斯溅射作为"学生",通过知识蒸馏实现两者的优势互补。整个框架分为两个主要阶段:

\textbf{阶段1:可控视频扩散模型预训练}
在大规模驾驶数据上训练一个以LiDAR几何条件为引导的视频扩散模型。该模型学习从稀疏的几何信息(LiDAR点云渲染)生成高质量、时间连贯的驾驶视频。

\textbf{阶段2:4DGS知识蒸馏优化}
利用预训练的扩散模型为新视点生成高质量的"伪真值"图像,以此作为监督信号指导4DGS模型的训练,从而提升其视点外推能力。

这种设计的核心优势在于:(1)扩散模型提供了强大的视觉先验和生成能力;(2)4DGS保持了实时渲染的效率;(3)知识蒸馏桥接了两者,实现了"鱼与熊掌兼得"。

\subsection{数据流与训练策略概览}

整个训练过程涉及多种类型的数据和处理流程:

\textbf{输入数据类型:}
\begin{itemize}
\item \textbf{RGB图像序列}:来自Waymo的5个相机(前、左前、右前、左、右),提供视觉真值
\item \textbf{LiDAR点云}:64线激光雷达数据,提供几何约束和深度监督
\item \textbf{物体轨迹}:3D边界框序列,用于动静分离和物体跟踪
\item \textbf{相机标定}:内外参数,确保几何一致性
\item \textbf{天空掩码}:通过Grounding DINO + SAM生成,用于天空区域的特殊处理
\end{itemize}

\textbf{训练数据流程:}
\begin{enumerate}
\item \textbf{几何条件生成}:将LiDAR点云投影并渲染为RGB-D条件图像
\item \textbf{扩散模型预训练}:使用RGB图像和LiDAR条件训练可控视频生成模型
\item \textbf{新视点轨迹生成}:为训练场景创建虚拟的换道、转弯等新轨迹
\item \textbf{伪真值生成}:扩散模型基于新轨迹的LiDAR条件生成高质量视频
\item \textbf{4DGS联合训练}:同时使用真实图像和扩散生成的伪真值训练4DGS
\end{enumerate}

\section{LiDAR几何条件的设计与使用}

\subsection{LiDAR数据的多层次利用}

LiDAR数据在我们的系统中发挥了三个关键作用:几何条件生成、深度监督和动静分离。这种多层次利用确保了几何一致性和训练稳定性。

\textbf{几何条件生成:}
我们将LiDAR点云渲染为像素级的几何条件,为扩散模型提供精确的3D结构约束。具体流程包括:

\begin{enumerate}
\item \textbf{点云聚合}:聚合$\pm\Delta t$帧内的LiDAR点云以增加密度
\item \textbf{点云着色}:通过相机图像为点云添加RGB信息
\item \textbf{可微分渲染}:将彩色点云渲染为RGB-D条件图像
\end{enumerate}

数学上,给定时刻$t$的LiDAR点云$\mathcal{P}_t = \{\mathbf{p}_i, \mathbf{c}_i\}_{i=1}^{N_t}$,我们首先进行时间聚合:

\begin{equation}
\mathcal{P}_{\text{agg}}(t) = \bigcup_{\tau=t-\Delta t}^{t+\Delta t} \mathcal{T}_{\text{ego}}(\tau \rightarrow t) \circ \mathcal{P}_\tau
\label{eq:lidar_aggregation}
\end{equation}

其中$\mathcal{T}_{\text{ego}}(\tau \rightarrow t)$是自车从时刻$\tau$到$t$的位姿变换。

然后通过可微分点渲染生成条件图像:

\begin{equation}
\mathbf{C}_{\text{lidar}}(u,v) = \sum_{i: \pi(\mathbf{p}_i) = (u,v)} w_i \cdot [\mathbf{c}_i; d_i; m_i]
\label{eq:lidar_condition_generation}
\end{equation}

其中$\pi(\cdot)$是3D到2D的投影函数,$w_i$是可见性权重,$d_i$是归一化深度,$m_i$是有效性掩码。

\textbf{深度监督:}
LiDAR提供的稀疏深度用于约束4DGS的几何重建:

\begin{equation}
\mathcal{L}_{\text{depth}} = \frac{1}{|\mathcal{M}_{\text{valid}}|} \sum_{(u,v) \in \mathcal{M}_{\text{valid}}} \rho(D_{\text{render}}(u,v) - D_{\text{lidar}}(u,v))
\label{eq:depth_supervision}
\end{equation}

其中$\mathcal{M}_{\text{valid}}$是有效LiDAR深度像素,$\rho(\cdot)$是Huber损失函数。

\subsection{动静分离的LiDAR处理策略}

为了支持动态场景建模,我们对LiDAR点云进行动静分离处理:

\textbf{静态背景点云:}
通过物体轨迹信息,将不在任何3D边界框内的点云归类为静态背景:

\begin{equation}
\mathcal{P}_{\text{bkgd}}(t) = \{\mathbf{p}_i \in \mathcal{P}_t : \forall j, \mathbf{p}_i \notin \mathcal{B}_j(t)\}
\label{eq:background_points}
\end{equation}

其中$\mathcal{B}_j(t)$是第$j$个物体在时刻$t$的3D边界框。

\textbf{动态物体点云:}
每个跟踪物体的点云单独处理:

\begin{equation}
\mathcal{P}_{\text{obj}}^{(j)}(t) = \{\mathbf{p}_i \in \mathcal{P}_t : \mathbf{p}_i \in \mathcal{B}_j(t)\}
\label{eq:object_points}
\end{equation}

这种分离策略的优势在于:(1)静态背景可以跨时间聚合,提高密度;(2)动态物体保持时间独立性,避免运动模糊;(3)支持物体级别的编辑和控制。

\section{天空区域的智能处理机制}

\subsection{基于Grounding DINO + SAM的天空分割}

天空区域在自动驾驶场景中具有特殊性:它缺乏明确的几何结构,且在不同视点下变化复杂。我们采用了基于视觉语言模型的智能分割策略:

\textbf{两阶段分割流程:}
\begin{enumerate}
\item \textbf{Grounding DINO检测}:使用文本提示"sky"检测天空区域的粗略边界框
\item \textbf{SAM精细分割}:基于检测框生成精确的像素级天空掩码
\end{enumerate}

具体实现中,我们使用以下策略确保天空检测的准确性:

\begin{verbatim}
# Sky detection with spatial constraints
boxes_mask = boxes_xyxy[:, 1] < 100  # Top 100 pixels only
valid_boxes = boxes_xyxy[boxes_mask]

# SAM segmentation with box prompts
sam_predictor.set_image(image_source)
transformed_boxes = sam_predictor.transform.apply_boxes_torch(valid_boxes, image_source.shape[:2])
masks, _, _ = sam_predictor.predict_torch(
    point_coords=None,
    point_labels=None, 
    boxes=transformed_boxes,
    multimask_output=False
)
\end{verbatim}

\textbf{天空区域的三种表示策略:}

1. \textbf{高斯天空模型}:使用独立的高斯基元集合表示天空,适用于天空区域较大的场景
2. \textbf{立方体贴图}:使用环境贴图表示天空,计算效率高但表达能力有限
3. \textbf{背景集成}:将天空直接集成到背景高斯模型中,适用于天空区域较小的场景

\subsection{天空损失的设计原理}

针对天空区域的特殊性质,我们设计了基于不透明度分布的天空损失:

\begin{equation}
\mathcal{L}_{\text{sky}} = \mathbb{E}_{(u,v) \in \mathcal{M}_{\text{sky}}} [-\log(1-\alpha(u,v))] + \mathbb{E}_{(u,v) \in \mathcal{M}_{\text{non-sky}}} [H(\alpha(u,v))]
\label{eq:sky_loss_design}
\end{equation}

其中$H(\alpha) = -\alpha \log \alpha - (1-\alpha) \log(1-\alpha)$是二元熵函数。这种设计鼓励:
- 天空区域的低不透明度($\alpha \rightarrow 0$)
- 非天空区域的确定性不透明度($\alpha \rightarrow 0$ 或 $\alpha \rightarrow 1$)

\section{训练流程的系统性设计}

\subsection{分阶段训练策略}

我们的训练策略采用了精心设计的分阶段方法,每个阶段都有明确的目标和数据需求:

\textbf{阶段1:扩散模型预训练(25-30 epochs)}

\textbf{数据需求:}
\begin{itemize}
\item RGB视频序列:$\mathbf{X} = \{\mathbf{x}_t\}_{t=1}^{T}$,$T=25$帧
\item LiDAR条件:$\mathbf{C} = \{\mathbf{c}_t\}_{t=1}^{T}$,RGB-D格式
\item 相机参数:内外参数用于几何一致性
\end{itemize}

\textbf{训练目标:}
\begin{equation}
\mathcal{L}_{\text{diffusion}} = \mathbb{E}_{t,\epsilon,\mathbf{x}_0,\mathbf{C}} \left[ \|\epsilon - \epsilon_\theta(\sqrt{\bar{\alpha}_t} \mathbf{x}_0 + \sqrt{1-\bar{\alpha}_t} \epsilon, \mathbf{C}, t)\|_2^2 \right]
\label{eq:diffusion_pretraining}
\end{equation}

\textbf{关键技术:}
- 条件丢弃(Condition Dropout):$p_{\text{drop}} = 0.1$,支持分类器自由引导
- 时空注意力:确保生成视频的时间连贯性
- 多尺度训练:支持不同分辨率的生成

\textbf{阶段2:4DGS基础训练(前7000次迭代)}

\textbf{数据需求:}
\begin{itemize}
\item 训练视点:$\mathcal{V}_{\text{train}}$,包含相机参数和真实图像
\item 彩色点云:用于高斯基元初始化
\item 物体轨迹:用于动静分离
\item LiDAR深度:用于几何监督
\end{itemize}

\textbf{训练目标:}
\begin{equation}
\mathcal{L}_{\text{base}} = \mathcal{L}_{\text{recon}} + \lambda_{\text{depth}} \mathcal{L}_{\text{depth}} + \lambda_{\text{sky}} \mathcal{L}_{\text{sky}}
\label{eq:base_training}
\end{equation}

\textbf{关键操作:}
- 自适应密化:每100次迭代进行一次,基于梯度和几何特征
- 球谐函数递增:每1000次迭代增加球谐函数阶数
- 不透明度重置:每3000次迭代重置低质量高斯基元

\textbf{阶段3:扩散蒸馏训练(7000-30000次迭代)}

\textbf{数据需求:}
\begin{itemize}
\item 原始训练数据:继续使用真实图像进行重建训练
\item 新视点轨迹:虚拟的换道、转弯等轨迹
\item 扩散生成的伪真值:高质量的新视点监督信号
\item 天空掩码:用于掩码引导和损失计算
\end{itemize}

\textbf{训练目标:}
\begin{equation}
\mathcal{L}_{\text{distill}} = \mathcal{L}_{\text{recon}} + \lambda_{\text{novel}} \mathcal{L}_{\text{novel}} + \lambda_{\text{reg}} \mathcal{L}_{\text{reg}}
\label{eq:distillation_training}
\end{equation}

\textbf{关键策略:}
- 渐进式采样:在特定迭代点(7000, 12000, 17000, 22000)进行扩散采样
- 引导强度衰减:从强引导($s=0.7$)逐渐减弱到弱引导($s=0.3$)
- 掩码引导:在15000次迭代后启用,只在渲染质量差的区域应用扩散监督

\subsection{新视点轨迹的生成策略}

为了训练4DGS的视点外推能力,我们需要生成多样化的新视点轨迹。这些轨迹模拟真实驾驶中的各种机动行为:

\textbf{横向平移轨迹}:模拟换道行为
\begin{equation}
\mathbf{T}_{\text{lateral}}(t) = \mathbf{T}_{\text{orig}}(t) \cdot \text{Translate}(d_{\text{lateral}} \cdot \mathbf{e}_y, 0, 0)
\label{eq:lateral_trajectory}
\end{equation}

\textbf{纵向偏移轨迹}:模拟跟车距离变化
\begin{equation}
\mathbf{T}_{\text{longitudinal}}(t) = \mathbf{T}_{\text{orig}}(t) \cdot \text{Translate}(0, d_{\text{longitudinal}}, 0)
\label{eq:longitudinal_trajectory}
\end{equation}

\textbf{高度变化轨迹}:模拟不同车辆高度
\begin{equation}
\mathbf{T}_{\text{height}}(t) = \mathbf{T}_{\text{orig}}(t) \cdot \text{Translate}(0, 0, d_{\text{height}})
\label{eq:height_trajectory}
\end{equation}

这些新轨迹的LiDAR条件通过几何变换自动生成,确保了条件的几何一致性。

\section{可控视频扩散模型的深度设计}

\subsection{条件控制机制的多层次设计}

我们的视频扩散模型采用了多层次的条件控制机制,确保生成内容的几何一致性和视觉质量:

\textbf{像素级几何条件}:LiDAR渲染的RGB-D图像提供逐像素的几何约束
\textbf{全局场景条件}:相机参数和场景元数据提供全局上下文
\textbf{时间条件}:帧索引和时间戳确保时间连贯性

条件编码过程可以表示为:

\begin{align}
\mathbf{h}_{\text{pixel}} &= \text{Conv2D}(\mathbf{C}_{\text{lidar}}) \\
\mathbf{h}_{\text{global}} &= \text{MLP}([\mathbf{K}; \mathbf{T}; t]) \\
\mathbf{h}_{\text{cond}} &= \text{CrossAttention}(\mathbf{h}_{\text{pixel}}, \mathbf{h}_{\text{global}})
\label{eq:condition_encoding_multilevel}
\end{align}

\subsection{时空注意力的创新设计}

为了确保生成视频的时间连贯性,我们设计了分解式时空注意力机制:

\begin{equation}
\text{SpatioTemporalAttention}(\mathbf{X}) = \text{TemporalAttention}(\text{SpatialAttention}(\mathbf{X}))
\label{eq:factorized_attention}
\end{equation}

这种分解设计的优势:
- 计算复杂度从$O(T^2H^2W^2)$降低到$O(TH^2W^2 + T^2HW)$
- 更好地建模空间和时间的不同特性
- 支持不同分辨率的高效处理

\section{训练自由引导的创新机制}

\subsection{双重引导策略}

我们提出的训练自由引导机制是StreetCrafter的核心创新之一。与传统的分类器自由引导不同,我们同时使用几何条件和当前4DGS渲染结果作为引导信号:

\begin{equation}
\epsilon_{\text{guided}} = \epsilon_{\text{uncond}} + w_{\text{geom}} \cdot (\epsilon_{\text{geom}} - \epsilon_{\text{uncond}}) + w_{\text{render}} \cdot (\epsilon_{\text{render}} - \epsilon_{\text{uncond}})
\label{eq:dual_guidance}
\end{equation}

其中:
- $\epsilon_{\text{geom}}$:基于LiDAR几何条件的噪声预测
- $\epsilon_{\text{render}}$:基于当前4DGS渲染结果的噪声预测
- $w_{\text{geom}}, w_{\text{render}}$:自适应调整的引导权重

\textbf{自适应权重调整:}
\begin{align}
w_{\text{geom}}(k) &= w_{\text{geom}}^{\text{init}} \cdot (1 + \cos(\pi k / K)) / 2 \\
w_{\text{render}}(k) &= w_{\text{render}}^{\text{init}} \cdot \exp(-k / \tau_{\text{decay}})
\label{eq:adaptive_guidance_weights}
\end{align}

这种设计确保了训练初期的强几何约束和后期的精细调优。

\subsection{掩码引导的精确控制}

在训练后期(>15000次迭代),我们引入掩码引导机制,只在渲染质量较差的区域应用扩散监督:

\begin{equation}
\mathbf{M}_{\text{guide}}(u,v) = \begin{cases}
1, & \text{if } \|\mathbf{I}_{\text{render}}(u,v) - \mathbf{I}_{\text{ref}}(u,v)\|_2 > \tau_{\text{error}} \\
0, & \text{otherwise}
\end{cases}
\label{eq:masked_guidance}
\end{equation}

这种策略的优势:
- 避免在已经收敛的区域进行不必要的修正
- 集中计算资源在最需要改进的区域
- 防止过度拟合扩散模型的生成偏好

\section{4DGS增强优化的系统性方法}

\subsection{多模态损失函数的协同设计}

我们设计了一个综合的多模态损失函数,平衡重建质量、几何一致性和新视点泛化:

\begin{align}
\mathcal{L}_{\text{total}} &= \mathcal{L}_{\text{recon}} + \lambda_{\text{novel}} \mathcal{L}_{\text{novel}} + \lambda_{\text{depth}} \mathcal{L}_{\text{depth}} \\
&\quad + \lambda_{\text{scale}} \mathcal{L}_{\text{scale}} + \lambda_{\text{sky}} \mathcal{L}_{\text{sky}} + \lambda_{\text{temporal}} \mathcal{L}_{\text{temporal}}
\label{eq:comprehensive_loss}
\end{align}

\textbf{重建损失}:多尺度感知损失组合
\begin{align}
\mathcal{L}_{\text{recon}} &= \mathcal{L}_{\text{L1}} + \lambda_{\text{ssim}} \mathcal{L}_{\text{SSIM}} + \lambda_{\text{lpips}} \mathcal{L}_{\text{LPIPS}} \\
\mathcal{L}_{\text{L1}} &= \|\mathbf{M} \odot (\mathbf{I}_{\text{render}} - \mathbf{I}_{\text{gt}})\|_1 \\
\mathcal{L}_{\text{SSIM}} &= 1 - \text{SSIM}(\mathbf{I}_{\text{render}}, \mathbf{I}_{\text{gt}}, \mathbf{M}) \\
\mathcal{L}_{\text{LPIPS}} &= \text{LPIPS}(\mathbf{M} \odot \mathbf{I}_{\text{render}}, \mathbf{M} \odot \mathbf{I}_{\text{gt}})
\label{eq:reconstruction_loss_detailed}
\end{align}

\textbf{新视点蒸馏损失}:基于扩散生成的伪真值
\begin{equation}
\mathcal{L}_{\text{novel}} = \|\mathbf{M}_{\text{upper}} \odot (\mathbf{I}_{\text{render}}^{\text{novel}} - \mathbf{I}_{\text{pseudo}})\|_1 + \lambda_{\text{novel\_ssim}} (1 - \text{SSIM}(\mathbf{I}_{\text{render}}^{\text{novel}}, \mathbf{I}_{\text{pseudo}}, \mathbf{M}_{\text{upper}}))
\label{eq:novel_distillation_loss}
\end{equation}

其中$\mathbf{M}_{\text{upper}}$排除图像上半部分,避免天空区域的不稳定性。

\textbf{时间一致性损失}:确保动态物体的平滑运动
\begin{equation}
\mathcal{L}_{\text{temporal}} = \sum_{i=1}^{N_{\text{obj}}} \sum_{t=1}^{T-1} \|\mathbf{T}_i(t+1) - \mathbf{T}_i(t)\|_{\text{SE(3)}}^2
\label{eq:temporal_consistency}
\end{equation}

\subsection{自适应密化的智能策略}

传统3DGS的密化策略在动态场景中面临挑战。我们提出了时间感知的自适应密化策略:

\begin{algorithm}
\caption{时间感知的4D高斯密化算法}
\label{alg:temporal_aware_densification}
\begin{algorithmic}[1]
\REQUIRE 当前高斯集合$\mathcal{G}$,时间窗口$[t-\Delta t, t+\Delta t]$
\ENSURE 更新后的高斯集合$\mathcal{G}'$

\STATE \textbf{Step 1: 时间聚合梯度统计}
\FOR{每个高斯基元 $g_i \in \mathcal{G}$}
    \STATE $\mathbf{g}_i^{\text{temporal}} = \frac{1}{|\mathcal{T}_{\text{visible}}|} \sum_{t \in \mathcal{T}_{\text{visible}}} \nabla_{\boldsymbol{\mu}_i} \mathcal{L}(t)$
\ENDFOR

\STATE \textbf{Step 2: 几何特征分析}
\FOR{每个高斯基元 $g_i \in \mathcal{G}$}
    \STATE 计算最大尺度:$s_{\max,i} = \max(\mathbf{s}_i)$
    \STATE 计算屏幕投影大小:$r_{\text{screen},i} = \text{ProjectedRadius}(\mathbf{s}_i, \mathbf{T}_{\text{cam}})$
\ENDFOR

\STATE \textbf{Step 3: 智能密化决策}
\FOR{每个高斯基元 $g_i \in \mathcal{G}$}
    \IF{$\|\mathbf{g}_i^{\text{temporal}}\| > \tau_{\text{grad}}$}
        \IF{$s_{\max,i} > \tau_{\text{scale}}$ AND $r_{\text{screen},i} > \tau_{\text{screen}}$}
            \STATE 执行分裂:$\boldsymbol{\mu}_{\text{new}} = \boldsymbol{\mu}_i \pm \frac{\mathbf{s}_i}{4}$
        \ELSE
            \STATE 执行克隆:$\boldsymbol{\mu}_{\text{new}} = \boldsymbol{\mu}_i + \mathcal{N}(0, \sigma_{\text{clone}}^2)$
        \ENDIF
    \ENDIF
    
    \IF{$\alpha_i < \tau_{\text{opacity}}$ OR $s_{\max,i} > \tau_{\text{prune}}$ OR $r_{\text{screen},i} > r_{\text{max}}$}
        \STATE 执行剪枝:$\mathcal{G} \leftarrow \mathcal{G} \setminus \{g_i\}$
    \ENDIF
\ENDFOR

\RETURN $\mathcal{G}'$
\end{algorithmic}
\end{algorithm}

\section{系统训练的完整数据依赖图}

\subsection{数据流向分析}

整个训练过程的数据依赖关系可以用以下流向图表示:

\textbf{原始数据 → 预处理数据:}
\begin{itemize}
\item Waymo TFRecord → RGB图像序列 + LiDAR点云 + 物体轨迹 + 相机标定
\item RGB图像 + Grounding DINO + SAM → 天空掩码
\item LiDAR点云 + RGB图像 → 彩色点云
\item 彩色点云 + 相机参数 → LiDAR条件图像(RGB-D)
\end{itemize}

\textbf{预处理数据 → 训练数据:}
\begin{itemize}
\item RGB序列 + LiDAR条件 → 扩散模型训练对
\item 彩色点云 + 物体轨迹 → 4DGS初始化
\item 原始轨迹 + 几何变换 → 新视点轨迹
\item 新视点轨迹 + LiDAR条件 → 扩散采样输入
\end{itemize}

\textbf{训练数据 → 监督信号:}
\begin{itemize}
\item 真实RGB图像 → 重建损失监督
\item LiDAR深度 → 几何一致性监督  
\item 扩散生成图像 → 新视点蒸馏监督
\item 天空掩码 → 天空区域损失监督
\end{itemize}

\subsection{训练调度的精确设计}

我们的训练调度采用了精心设计的多阶段策略:

\textbf{迭代0-500:预热阶段}
- 仅使用重建损失,建立基础几何结构
- 较低的学习率:$\eta_{\text{pos}} = 1.6 \times 10^{-4}$
- 禁用密化操作,避免早期不稳定

\textbf{迭代500-7000:基础训练阶段}  
- 启用完整的重建损失和正则化项
- 自适应密化:每100次迭代
- 球谐函数递增:每1000次迭代从0阶增加到3阶

\textbf{迭代7000+:扩散蒸馏阶段}
- 在特定迭代点进行扩散采样:$\{7000, 12000, 17000, 22000\}$
- 渐进式引导强度:$s \in \{0.7, 0.5, 0.3, 0.1\}$
- 新视点采样概率:$p_{\text{novel}} = 0.1$

\textbf{迭代15000+:掩码引导阶段}
- 启用掩码引导,精细化训练
- 降低密化频率,稳定几何结构
- 引入时间一致性约束

\section{多模态正则化的协同设计}

\subsection{几何一致性约束}

\textbf{LiDAR深度约束}:
利用LiDAR提供的稀疏但准确的深度信息约束4DGS的几何重建:

\begin{equation}
\mathcal{L}_{\text{depth}} = \frac{1}{|\mathcal{M}_{\text{lidar}}|} \sum_{(u,v) \in \mathcal{M}_{\text{lidar}}} \text{SmoothL1}(D_{\text{render}}(u,v), D_{\text{lidar}}(u,v))
\label{eq:lidar_depth_constraint}
\end{equation}

\textbf{物体边界约束}:
确保动态物体的高斯基元不会泄漏到其边界框外:

\begin{equation}
\mathcal{L}_{\text{bbox}} = \sum_{i=1}^{N_{\text{obj}}} \mathbb{E}_{\mathbf{p} \notin \mathcal{B}_i} [\alpha_i(\mathbf{p})] + \mathbb{E}_{\mathbf{p} \in \mathcal{B}_i} [H(\alpha_i(\mathbf{p}))]
\label{eq:bbox_constraint}
\end{equation}

\subsection{尺度和形状正则化}

\textbf{各向异性约束}:
防止高斯基元退化为过度拉伸的椭球体:

\begin{equation}
\mathcal{L}_{\text{anisotropy}} = \mathbb{E}_{g \in \mathcal{G}} \left[ \frac{\max(\mathbf{s}_g)}{\min(\mathbf{s}_g)} - 1 \right]^2
\label{eq:anisotropy_regularization}
\end{equation}

\textbf{紧凑性约束}:
鼓励高斯基元保持紧凑的形状:

\begin{equation}
\mathcal{L}_{\text{compact}} = \mathbb{E}_{g \in \mathcal{G}} \left[ \|\mathbf{s}_g\|_2^2 \right]
\label{eq:compactness_regularization}
\end{equation}

\section{理论分析与性能保证}

\subsection{收敛性的理论保证}

\begin{theorem}[多模态损失收敛性]
在以下条件下:
\begin{enumerate}
\item 扩散模型生成质量有界:$\mathbb{E}[\|\mathcal{D}_\phi(\mathbf{z}, \mathbf{C}) - \mathbf{I}_{\text{true}}\|_2] \leq \epsilon_D$
\item 4DGS表示能力充分:存在$\mathcal{G}^*$使得表示误差有界
\item 学习率满足标准条件:$\sum \eta_k = \infty$,$\sum \eta_k^2 < \infty$
\end{enumerate}
优化过程几乎必然收敛到$\epsilon_D$-邻域内的最优解:
\begin{equation}
\lim_{k \to \infty} \mathbb{E}[\mathcal{L}_{\text{total}}^{(k)}] \leq \mathcal{L}^* + C \cdot \epsilon_D
\end{equation}
\end{theorem}

\subsection{计算复杂度的全面分析}

\textbf{扩散模型复杂度:}
- 预训练:$O(E \cdot B \cdot T \cdot K \cdot H \cdot W \cdot D_{\text{model}})$
- 推理采样:$O(K \cdot T \cdot H \cdot W \cdot D_{\text{model}})$

\textbf{4DGS复杂度:}
- 渲染:$O(N_g \cdot H \cdot W \cdot \log N_g)$
- 密化:$O(N_g \cdot D_{\text{gaussian}})$
- 梯度计算:$O(N_g \cdot H \cdot W)$

\textbf{总体训练复杂度:}
\begin{equation}
\mathcal{O}_{\text{total}} = O(I \cdot (N_g \cdot H \cdot W + N_{\text{sample}} \cdot K \cdot T \cdot H \cdot W \cdot D_{\text{model}}))
\label{eq:total_complexity}
\end{equation}

其中$I$是总迭代数,$N_{\text{sample}}$是扩散采样频率。

\section{本章小结}

本章从系统设计的整体视角详细阐述了StreetCrafter的核心方法论。我们首先介绍了双阶段知识蒸馏的整体框架和设计理念,然后深入分析了LiDAR几何条件的多层次利用、基于Grounding DINO + SAM的智能天空处理、以及精心设计的分阶段训练策略。

训练自由引导机制是我们的核心创新,通过双重引导和掩码控制实现了精确的知识迁移。多模态损失函数的协同设计确保了重建质量、几何一致性和新视点泛化能力的平衡。自适应密化算法针对动态场景进行了专门优化,支持时间感知的几何表示。

理论分析证明了方法的收敛性和复杂度特性,为实际应用提供了理论指导。该方法论框架为解决4DGS在视点外推任务上的挑战提供了系统性的解决方案,下一章将详细介绍具体的工程实现和系统架构。