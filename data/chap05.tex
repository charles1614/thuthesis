% !TeX root = ../postdoc-thesis.tex

\chapter{总结与展望}

\section{全文总结}

本论文针对自动驾驶领域高保真场景重建中的视点外推挑战,系统性地研究了基于扩散先验的4D Gaussian Splatting增强方法,提出了ReGen系统框架。本文的主要研究内容和贡献可以从理论方法、系统架构、工程实现和应用价值四个维度进行总结。

在理论方法层面,本文提出了基于扩散先验的4D Gaussian Splatting增强框架,实现了生成先验与几何表示的有机融合。通过LiDAR几何条件的精心设计,使扩散模型能够在保持生成多样性的同时确保几何一致性。训练自由引导机制和自适应采样调度算法的提出,有效解决了扩散监督训练中的稳定性问题。掩码引导策略通过选择性地在渲染质量较差的区域应用扩散监督,提高了训练效率并避免了过度监督导致的模型退化。多模态正则化框架通过深度一致性、光流平滑性和感知损失的协同作用,确保了重建场景的几何准确性和时间连贯性。

在系统架构层面,本文提出了五层解耦的系统架构,包含数据抽象层、几何表示层、神经渲染层、扩散推理层和扩散监督层。这种分层设计有效降低了系统复杂度,提高了可维护性和可扩展性。数据抽象层负责输入数据的统一格式化和预处理。几何表示层采用静态背景、动态物体和环境映射的三元组合表示,支持大规模城市级场景的高效建模。神经渲染层实现了基于tile的光栅化引擎,实现了实时渲染性能。扩散推理层封装了基于Diffusion Transformer架构的视频扩散模型,集成了滑动窗口机制和键值缓存技术以支持长序列生成。扩散监督层协调两个模型的联合优化,实现了内存高效的异步计算流水线。

在工程实现层面,本文基于Wan2.2-I2V-A14B的Diffusion Transformer架构进行了系统集成与优化。混合专家(MoE)机制通过基于信噪比的自适应专家路由,使高噪声专家负责整体布局和动态构建,低噪声专家专注于细节优化,显著提升了视频生成的质量和效率。针对内存管理和并行计算的挑战,系统实现了分层内存管理策略和异步计算流水线。融合FSDP和Ulysses序列并行的分布式训练架构通过参数分片、梯度压缩和动态负载均衡策略,支持在多GPU集群上进行大规模模型的高效训练。灵活的多模式推理引擎能够适应不同硬件配置和部署场景的需求。

在应用价值层面,本文研究成果已实现量产落地。基于场景重建技术生成的corner case图像数据已应用于自动驾驶感知和规划算法的训练与测试,有效提升了算法在长尾场景中的鲁棒性和泛化能力。该成果不仅降低了数据采集成本,更系统性地覆盖了测试空间中的关键区域,显著提升了系统的安全性和可靠性。

由于内部数据涉及保密要求,本文使用Waymo Open Dataset作为公开数据集进行实验验证。在视点外推任务上相比现有方法在图像质量和感知质量上均取得显著提升,同时保持实时渲染效率。消融实验验证了各技术组件的有效性,定性分析展示了系统在复杂城市场景中的渲染质量。

总体而言,本论文建立了从理论方法到工程实现的完整技术体系,并通过量产落地验证了方法的实用价值,为高保真场景重建技术在自动驾驶领域的深入应用提供了技术支撑。

\section{未来展望}

尽管本文在场景重建技术方面取得了系统性成果,并实现了在corner case生成方面的成功落地,但自动驾驶仿真技术的发展仍有广阔的探索空间。基于当前研究基础,本文展望以下几个具有重要战略价值的研究方向。

\subsection{闭环仿真系统的构建}

当前的场景重建技术主要关注静态场景的高保真渲染,缺乏对自动驾驶系统行为的动态响应能力。未来需要构建真正的闭环仿真系统,使重建的虚拟环境能够根据自车的决策和行动实时生成相应的场景变化。这种闭环能力将使仿真系统不仅能够重现已有场景,更能够模拟自动驾驶系统与环境的交互过程,包括其他交通参与者对自车行为的反应、环境状态的动态演化等。

闭环仿真系统的构建需要解决场景动态响应建模、物理一致性保证、实时交互能力和长时程稳定性等关键技术挑战。通过构建闭环系统,可实现对感知、规划和控制算法的全面测试,对于发现系统性问题和评估极端情况下的系统行为具有重要价值。

\subsection{强化学习驱动的行为建模}

强化学习技术的引入对于提升交通参与者行为的真实性和多样性具有重要意义。通过强化学习训练的智能体能够自主决策、适应环境变化,在战术层面学习复杂驾驶行为,在交互层面通过多智能体强化学习建模车辆间的博弈关系,在对抗层面主动生成具有挑战性的corner case场景。将强化学习与场景重建相结合,可构建智能的仿真环境,通过主动式测试方法高效发现系统缺陷,显著提升测试覆盖率。

\subsection{视频生成模型的深度应用}

视频生成模型具有强大的时序建模能力和生成先验知识,可在场景预测推演、跨模态转换、数据增强合成和交互式编辑等方面发挥重要作用。特别是对于长尾场景,生成模型可在保证真实性前提下大量合成数据。随着视频生成技术的快速进步,未来模型将能够生成更长时长、更高质量的视频内容,达到更高的保真度上限。通过将视频生成模型与物理引擎、行为模型深度集成,可构建兼具视觉真实性和物理准确性的混合仿真系统。

\subsection{技术融合与系统优化}

未来的自动驾驶仿真系统需要深度融合场景重建、闭环仿真、强化学习和视频生成等技术,构建模块化、可扩展的智能仿真平台。在理论层面,需建立统一的数学框架描述场景生成、交互建模和行为学习的关系;在工程层面,需开发高效的算法实现和分布式计算架构,实现海量场景的快速仿真。

\section{结束语}

自动驾驶技术正处在向完全自主驾驶跨越的关键阶段。本论文在场景重建方法论、系统架构和工程实现方面取得了系统性进展,并通过量产落地验证了实用价值。然而,从单一场景重建到闭环智能仿真,从静态数据生成到动态交互建模,仍有大量挑战性问题有待解决。强化学习驱动的行为建模和视频生成模型的深度应用,将使仿真系统达到更高的上限。

随着生成式AI技术的快速发展和计算能力的持续提升,未来的自动驾驶仿真系统将实现从"重现真实"到"超越真实"的跨越,不仅高保真再现已知场景,更能智能创造挑战性测试场景,主动发现系统潜在问题。这种能力将成为实现L5级自主驾驶不可或缺的技术基础。

本文研究工作是这一长期愿景的阶段性成果,希望为学术界和工业界的持续探索提供有价值的参考,共同推动场景重建和智能仿真技术向更高水平发展。

