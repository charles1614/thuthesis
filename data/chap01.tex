% !TeX root = ../thuthesis-example.tex

\chapter{引言}

随着自动驾驶技术向L5级完全自主驾驶的演进,高保真场景重建已从辅助感知的工具转变为支撑整个自动驾驶研发体系的核心技术基础。在这一历史性转变过程中,传统的场景理解方法正面临前所未有的挑战,而新兴的神经场景表示技术则为突破这些瓶颈提供了新的可能。本章将从自动驾驶技术发展的宏观视角出发,深入分析高保真场景重建的战略价值,系统阐述神经场景表示技术的演进轨迹,重点探讨当前技术面临的核心挑战,并明确本论文的研究动机和主要贡献。

\section{高保真场景重建的战略意义与技术挑战}

自动驾驶技术的发展历程清晰地展现了从简单辅助驾驶到完全自主驾驶的技术演进路径。在这一演进过程中,场景理解技术的需求发生了根本性的变化。早期的L2和L3级别系统主要依赖于高度抽象化的环境表示方法,包括2D和3D边界框、车道线检测以及交通标志识别等符号化信息处理技术\cite{chen2025snerf}。这种稀疏化的场景表示方法在结构化程度较高的高速公路环境中表现相对可接受,但在面对复杂多变的城市驾驶场景时,其局限性变得日益明显。

L5级完全自主驾驶技术的核心挑战在于如何稳健地处理现实世界中的"长尾问题"。这些长尾场景包括但不限于恶劣天气条件下的驾驶、复杂交通状况的应对、基础设施异常情况的处理等。传统的数据驱动方法在面对这些罕见但安全关键的驾驶场景时,往往受到数据稀缺性和数据采集成本高昂的双重制约\cite{yan2024street}。据业界统计,要通过真实世界的道路测试来覆盖99.9\%的安全关键场景,需要累计行驶超过110亿英里的里程,这在经济成本和时间投入上都是难以承受的。

在这样的技术背景下,高保真场景重建技术的价值定位发生了战略性的转移。其角色已经从提升直接感知能力的辅助工具,转变为赋能大规模仿真测试的核心技术基础。这种根本性的转变对技术标准提出了更加严苛的要求。首先是几何精度要求的显著提升,需要从传统的厘米级精度提升到毫米级精度,以支持更加精确的碰撞检测和路径规划算法。其次是视觉保真度的大幅改善,需要从功能性表示提升到照片级真实感,以支持感知算法的闭环测试验证。此外,还需要具备良好的时间连贯性,能够从静态场景快照扩展到动态序列建模,以支持复杂驾驶行为的高保真仿真。最后,系统还必须具备强大的视点可控性,能够从固定视角的观测扩展到任意视点的渲染,以支持多样化驾驶策略的全面测试。

视点外推能力是高保真场景重建技术面临的最核心挑战之一。从技术本质上看,视点外推问题可以理解为在训练视点分布范围之外的新视点上进行高质量渲染的技术难题。这一挑战的根源在于几何表示方法的固有不完整性和有效先验知识的缺失。传统的显式表示方法,如点云和三角网格,在训练数据覆盖不到的空间区域往往缺乏有效的插值机制,而隐式表示方法虽然在理论上具有更好的插值能力,但其巨大的计算开销使其难以满足实时仿真应用的性能要求。

从信息论的角度来理解,视点外推问题本质上是一个条件生成问题,即如何在给定有限训练观测的条件下,对未见视点进行合理的场景内容推断。传统方法的主要局限在于其学习到的场景表示过于尖锐和特化,缺乏足够的泛化能力来处理训练分布之外的视点查询。这种局限性在实际应用中表现为渲染质量随视点偏离程度的急剧下降,严重制约了高保真仿真系统的实用价值。

\section{神经场景表示技术的演进与现状分析}

神经场景表示技术的发展历程体现了从隐式表示到显式表示的重要技术范式转变。Neural Radiance Fields(NeRF)作为隐式神经表示的典型代表,通过多层感知机网络学习连续的体积密度和辐射场函数,实现了前所未有的场景重建质量\cite{mildenhall2021nerf}。NeRF的核心创新在于将场景建模为一个连续的五维函数,该函数能够将三维空间位置和二维观测方向映射为体积密度和辐射颜色。这种连续函数表示的优势在于其天然的可微分性质,使得模型能够直接从二维图像监督中学习三维场景结构,无需额外的几何先验知识。

尽管NeRF在静态场景重建方面取得了突破性进展,但其在处理大规模动态场景时面临三个根本性的技术挑战。首先是计算复杂度问题,每个像素的渲染过程需要沿光线方向进行密集采样并执行大量的神经网络前向推理,导致渲染速度远远低于实时应用的要求。其次是可扩展性限制,单一神经网络的表达容量有限,难以有效表征广阔的城市级场景。虽然Block-NeRF和Mega-NeRF等方法通过空间分割策略在一定程度上缓解了这一问题,但也引入了模型管理复杂性和边界不连续性等新的技术挑战\cite{blocknerf2022,meganeRF2022}。第三个挑战是动态建模的困难,NeRF的静态场景假设与动态驾驶环境的本质需求存在根本性矛盾。

3D Gaussian Splatting(3DGS)的出现标志着神经场景表示技术从隐式到显式的重要范式转变\cite{kerbl2023gaussian}。3DGS采用显式的三维高斯基元集合来表示场景,每个高斯基元通过位置、协方差矩阵、不透明度和颜色等参数进行完整描述。这种显式表示的最大优势在于渲染效率的显著提升,通过可微分的点splatting技术,3DGS能够实现超过100帧每秒的实时渲染性能,为实际应用部署奠定了坚实的技术基础。

然而,3DGS在视点外推任务上存在根本性的技术局限。这种局限源于显式表示方法的本质特性:3DGS在训练过程中主要学习如何合理排布高斯基元,使其从训练视点观察时能够产生正确的视觉效果,但缺乏强大的几何或语义先验来约束全局场景的一致性。这导致模型容易对特定的训练视点分布产生过拟合现象。与NeRF的连续函数表示不同,3DGS的离散基元表示在训练数据覆盖不足的空间区域缺乏有效的插值机制,当查询视点远离训练分布时,容易出现几何不一致和明显的视觉伪影。

更为关键的是,3DGS缺乏关于真实世界场景结构的内在先验知识,无法在数据稀疏区域进行合理的场景内容补全。实验分析表明,3DGS的视点外推性能与视点偏离距离呈现明显的指数衰减关系,这种性能退化严重限制了其在需要广泛视点覆盖的仿真应用中的实用价值。

\section{扩散模型的生成先验与融合机遇}

扩散模型通过在大规模数据集上的预训练,已经内隐地学习了丰富的视觉先验知识\cite{ho2020denoising}。这些先验知识体现在模型对真实世界视觉模式的深度理解上,包括几何结构的合理性、光照效果的真实性、材质属性的一致性以及物理约束的遵循等多个方面。从贝叶斯学习的理论角度来看,扩散模型实际上学习了数据分布的梯度场,这种梯度场包含了关于数据分布结构的丰富信息,为指导三维场景的生成和优化提供了强大的理论基础。

分数蒸馏采样(Score Distillation Sampling, SDS)技术的提出为将二维扩散先验有效应用于三维内容生成提供了重要的理论框架\cite{dreamfusion2022}。SDS的核心思想是将预训练的二维扩散模型转化为可微分的损失函数,通过梯度反传的方式指导三维表示的优化过程。这种方法的理论优势在于能够将扩散模型学习到的丰富视觉先验直接迁移到三维场景重建任务中,为解决视点外推等关键技术挑战提供了新的思路。

将扩散模型的生成先验迁移到三维表示的过程本质上是一个知识蒸馏问题。在这个框架中,预训练的扩散模型充当教师网络的角色,为三维表示的学习提供高质量的监督信号。这种知识蒸馏范式的理论优势体现在三个方面:首先是先验知识的有效迁移,能够将扩散模型在大规模数据上学习到的视觉先验迁移到三维表示中;其次是泛化能力的显著增强,通过生成先验的引导,三维表示在未见视点上的表现能够得到明显改善;最后是效率与质量的良好平衡,既保持了三维表示的渲染效率优势,又提升了生成内容的视觉质量。

当前融合神经场景表示与生成模型的方法主要可以归纳为三种技术范式。数据机器范式通过离线生成额外的训练数据来改善模型性能,其代表性工作包括DriveDreamer4D等。在线修复范式将生成模型作为实时修复器嵌入训练循环,典型代表是ReconDreamer等方法。知识蒸馏范式采用两阶段的训练流程实现知识迁移,StreetCrafter等工作属于这一类别。每种范式都有其独特的技术优势,但也存在相应的局限性,为本论文的技术创新提供了明确的改进空间。

\section{研究动机与核心贡献}

基于对现有技术发展状况的深入分析,本论文的研究动机可以概括为三个层面的技术需求。首先是理论层面的需求,现有方法缺乏统一的理论框架来分析不同融合范式的优劣,对知识蒸馏过程的收敛性和稳定性也缺乏严格的理论保证。其次是算法层面的需求,现有的引导机制相对简单,无法充分利用多模态信息;采样策略缺乏自适应性,难以根据训练状态进行动态调整;正则化机制不够完善,难以有效平衡多个优化目标。最后是系统层面的需求,缺乏端到端的优化框架使得组件间的协同效应未能得到充分发挥,内存和计算效率有待提升,标准化的评估框架也亟待建立。

针对这些技术需求,本论文提出了一种改进的知识蒸馏框架,主要贡献体现在四个方面。第一个贡献是训练自由引导机制的设计,提出了一种新颖的引导策略,能够同时利用几何条件和当前渲染状态作为引导信号,实现更加精确和稳定的知识迁移过程。第二个贡献是自适应采样调度算法的开发,设计了基于训练状态感知的自适应采样调度机制,能够根据4DGS模型的收敛情况动态调整扩散采样的频率和强度,显著提升了训练效率和蒸馏质量。第三个贡献是多模态正则化框架的构建,提出了综合考虑几何一致性、时间连贯性和视觉质量的正则化体系,通过多个正则化项的协同作用确保蒸馏过程的稳定性和最终结果的质量。第四个贡献是分层系统架构的设计,构建了五层解耦的系统架构,实现了复杂算法的模块化分解和高效实现,为大规模系统的开发和维护提供了良好的技术基础。

这些技术贡献不仅在理论上具有创新性,更重要的是在实际应用中展现出显著的性能优势。通过系统性的实验验证,本论文提出的方法在视点外推任务上相比现有最先进方法取得了显著的性能提升,为高保真场景重建技术在自动驾驶仿真中的实际应用奠定了坚实的技术基础。

\section{论文组织结构}

本论文的组织结构遵循从理论分析到实践验证的逻辑脉络。第二章系统回顾相关工作,深入分析神经场景表示方法、扩散模型在三维生成中的应用以及两者融合的最新进展,为本文的技术贡献提供充分的背景铺垫和对比基准。第三章详细阐述本文提出的核心方法论,包括问题的数学形式化描述、可控视频扩散模型的设计原理、知识蒸馏框架的构建方法以及训练自由引导机制的理论分析。第四章介绍系统的架构设计和工程实现细节,涵盖分层架构的设计原理、高性能渲染引擎的实现方案、内存优化策略以及分布式训练的技术方案。第五章展示全面的实验结果和深入分析,包括在多个标准数据集上的定量和定性评估、详细的消融实验分析以及与现有方法的系统性对比。第六章总结全文的主要贡献,客观分析方法的局限性,并对未来的研究方向进行前瞻性展望。
